%! Author = dominikfischer
%! Date = 2026-02-11
% Preamble
% Die erste (unkommentierte) Zeile im Dokument legt immer die Dokumentklasse fest
\documentclass[english, titlepage]{scrreprt}


% Präambel
% Einbinden von zusätzlichen Paketen.
% Solche Dateien werden "style files" genannt mit Typ sty.
% Im Folgenden wird die Datei 'pakete.sty' eingebunden.
% mit 'pakete.sty' wird auch das Paket cleveref eingebunden; dieses sollte als letztes eingebunden werden
\usepackage{listings}
\usepackage{physics}
\usepackage{pakete}
\usepackage[Draft]{Praktikum}
% dieses file definiert eigene listings-styles
\usepackage{code_sources}
\usepackage{chngcntr}
\usepackage{rotating}
\usepackage{subcaption}
\usepackage{booktabs}
\usepackage[version=4]{mhchem}


% Abbildungen, Pfade zu den Verzeichnissen, welche die Abbildungen enthalten
\graphicspath{{figs/matplotlib/}{figs/kafe2/}{figs/material/}{figs/}}

% Literatur
\addbibresource{mybib.bib}

% diese zusätzlichen Setups sollten in ein eigenes "style file" ausgelagert werden
% Deutsch als locale, wenn die Sprache Deutsch (ngerman) ist.


% Titel, Autor und Datum
\title{SmoDa Final Project}
\subtitle{Implementation of BFGS algorithm for likelihood optimisation}
\date{\today}
\author{\firstAuthor\\
Matrikelnummern: \immatriculation\\
E-Mail: \mailref{\firstMail}}

% Draft version - add the word DRAFT on the cover pages
\ifthenelse{\equal{\ThesisVersion}{Draft}}{%
    \usepackage{background}
    \backgroundsetup{contents=DRAFT, color=blue!30}
    \addto\extrasngerman{\backgroundsetup{contents=ENTWURF, color=blue!30}}
    \addto\extrasgerman{\backgroundsetup{contents=ENTWURF, color=blue!30}}
    \ifoot[\today{} \thistime]{\today{} \thistime}
}{
}

\usepackage{definitions}
\usepackage{praktikum_defs}

% Dokumentanfang
\begin{document}
    % \definechangesauthor[name=Dominik]{DF}
    % \definechangesauthor[name=Melis]{M}

    \maketitle
    \makeglossaries

    \pagenumbering{roman}
    \tableofcontents
    \clearpage
    \pagenumbering{arabic}


    \chapter{Demo Report}\label{ch:demo-report}


    \section{Introduction}\label{sec:introduction}
% Start your report by introducing the problem... Also, don't forget citations, e.g.~\cite{ParticleDataGroup:2024cfk}
    %! Author = dominikfischer
%! Date = 2026-02-10
This project is aimed to program one part of a fitting framework.
The implemented part should find the minimum/maximum of a provided likelihood function.
This is done by optimisation.

\paragraph{General}
There are different numerical methods on how to find such a minimum.
One of these possibility are the gradient descent or hill-climb methods.
The basic idea of these algorithms is to follow the direction of largest descent (largest decrease of the value of the cost function) until a minimum is reached.
There the approximation to such a \textbf{local} minimum is enhanced iteratively.
On each iteration the direction of descent is evaluated and searched for minimum to improve the previous result.
This is repeated until a required accuracy is reached or some other stopping condition is fulfilled.

\paragraph{Aim}
The Aim of this is to implement such an algorithm (in this case it is the BFGS Algorithm) and test it on two different (example) likelihood functions.
First it is necessary to implement these likelihoods or to be more accurate their negative-log-likelihood.
As constant values won't change the values of derivatives all terms within these nll-function which are either purely constant or do not depend on the parameters we are interested in will be dropped.
The first (basic) test will be performed on a gaussian distribution.
The likelihood itself could be defined here as
\begin{equation}
    \likelihood(\vec{\theta}) = \prod_{j}^{N}\frac{1}{(2\pi)^3 \sqrt{\det{V}}}\exp\left( -\frac{1}{2}(\vec{\theta}  - \vec{\mu}_\text{j})^\text{T}V^{-1}(\vec{\theta}  - \vec{\mu}_\text{j})\right)\label{eq:likelihood:gauss}
\end{equation}
These could be written as the negative-log-likelihood when dropping all constant terms.\footnote{Please keep in mind here that this representation is effectively a product over gaussian density functions for each data point (which is a vector consisting of six numbers)}
\begin{equation}
    \nllf(\vec{\theta}) = \sum_{j}^{N} (\vec{\theta} - \vec{\mu}_\text{j})^\text{T}V^{-1}(\vec{\theta} - \vec{\mu}_\text{j})\label{eq:nll:gauss}
\end{equation}
This is implemented by evaluating the log pdf for each data vector and then summing over these.
For demonstration purposes a random dataset is generated from this distribution first.
For the second part the rosenbrock function should be optimised.
The negative log-likelihood of the rosenbrock function\footnote{This is the rosenbrock function itself.} is given by
\begin{equation}
    \nllf(\vec{\theta}) = \sum_{i=1}^{5} \left(100\cdot\left( \theta_{\text{i} + 1} - \theta_\text{i}^2\right )^2 + (1 - \theta_\text{i})^2 \right )\label{eq:intro:rosen}
\end{equation}
% TODO: implement and report the parts for the rosenbrock function.



    %! Author = dominikfischer
%! Date = 2026-02-10


\section{Implementation of the Optimisation}\label{sec:implementation}

\subsection{General Aspects}\label{ssec:general_optimum}
%! Author = dominikfischer
%! Date = 2026-02-10
The BFGS Algorithm and our implementation is following the general form of a gradient descent algorithm.
So it is necessary to first introduce the general structure.

\paragraph{Gradient Descent}
A minimum or at least a critical point of \(f\) will fulfill \(\grad f = 0\).\footnote{For this necessary condition see e.g.\cite[14-23]{Nocedal}}
We will do so by iteratively improving an already known approximation to the optimum.
On each iteration we will search a for new local minimum starting from our currently best value along a direction \(\vec{p}\) (see also\cite{nonLinOptimerung, Nocedal}).

At the start of each iteration a suitable search direction is computed to search for a minimum, but first check the gradient at the best current approximation and stop the iteration if a stopping condition is already fulfilled.
Subsequently, a line search will be done to estimate the optimal step size.
This is an optimisation algorithm which tries to minimise a function along a line in (parameter) space.
% TODO: discuss the implementation used for the line search.
We will take a closer look at this in \cref{ssec:impl:line} and discuss also the implementation used.
These will optimise the value of \(f(\vec{x}_0 + \alpha\cdot\vec{p})\) in the positive definite parameter \(\alpha\), which is called the \enquote{step size}.
Only an approximate line search will be used as trying more search directions is more effective than using the absolutely best step size for each direction\cite{Nocedal, nonLinOptimerung}.
After estimating the step size, we could use this to perform the proper iteration step
\begin{equation}
    \vec{x}_{\text{k+1}} = \vec{x}_\text{k} + \alpha^\text{(k)}\cdot\vec{p}_\text{k}\label{eq:linesearch:step}
\end{equation}
The indices of the parameters \(\vec{x}\) and directions \(\vec{p}\) and the superscript of the step size \(\alpha\) indicate the iteration in this update formula.
The provided code base will do exactly these steps, but these are implemented in separate functions.
Their implementations will be discussed together with particular implemented algorithms in the upcoming sections \cref{ssec:bfgs,ssec:impl:line}.

\paragraph{search direction}
By definition the direction of the gradient of a (scalar) function is the direction of maximum descent of the functions values and so points locally in the direction of the maximum.
Therefore, to search in the direction of the minimum we need a component opposite to the gradient, or more formal their angle must be \(> \SI{90}{\degree}\).
Which is equivalent to the condition for the inner product \(\vec{p}^\text{T}\grad f < 0\)\footnote{Derivations of these aspects could be found at\cite{nonLinOptimerung,Nocedal}}.
But in opposite to the gradient of our function which could be calculated at least numerically if the function is differentiable\footnote{which is always the case for a smooth function} we could not directly calculate the search direction\footnote{meaning: it is not uniquely defined.}.
On the other hand, we know that the gradient\footnote{The implementations will calculate the gradient (and the hessian) by numerical derivatives. A central finite difference approach is chosen. Information on calculating the coefficients used by this method could be found at\cite{Nocedal, nonLinOptimierung, findiff, numdifftoolsguides}.} points locally to a maximum.
Therefore, we need a mapping from the gradient to the search direction.
This mapping could be chosen linear, say more accurately an endomorphism.
Such an endomorphism could always be presented by a \(N\times N\) matrix.
Thus, we could always write our search direction as
\begin{equation}
    \vec{p} = -B\cdot\grad f\label{eq:general:search:direction}
\end{equation}
In \cref{eq:general:search:direction} we use the \textbf{positive definite} matrix \(B\) and the matrix-vector-product to perform the mapping from the computed gradient to the search direction.
By taking the inner product of the search direction with the gradient and using the positive definiteness of \(B\) one immediately sees that the condition for searching for a minimum is fulfilled.
The choice of a positive definite matrix, e.g.\ the Hesse matrix or it's inverse is quite natural, as the Hesse matrix at an local minimum is (at least) positive semidefinite\cite[15, 22]{Nocedal}
But when talking about the search direction one must always keep in mind that it will only lead \textbf{locally} to a minimum.
When implementing the mapping from the gradient to the search direction we decided to not normalise the search direction or the gradient to unit length.

\paragraph{Examples}
In the simplest possible case, this matrix could be chosen to be the identity \(B = \identity{N}\).
This is the \enquote{traditional} gradient descent or hill climbing algorithm.
Another choice which could be seen from the derivation in \cref{ssec:procedure:newton:derivation} and \cref{eq:newton:iteration} is to choose the inverse of the Hessian of \(f\) or at least an approximation to it.
This particular choice will lead to the Newton and Quasi-Newton procedures for minimisation\cite{nonLinOptimerung,Nocedal}.
Such a choice is the BFGS algorithm which will be implemented as the primary contribution to the project.
The BFGS algorithm is a so called quasi-newton algorithm.
So it could be necessary to talk first about the newton algorithm in \cref{ssec:procedure:newton:derivation}.

\paragraph{Conclusion}
The past paragraphs could be summarised to a recursive rule for iteratively improving our approximation of the optimum of \(f\).
In each iteration we will start from some previously best value and calculate a search direction with respect to this value.
Next, we will perform a (short) line search to get the approximately best step size to achieve the largest possible improvement onto the function \(f\).
This new value for \(\vec{x}\) will be used as the starting value for the next iteration.
So the updating in each iteration step could be summarised by \cref{eq:linesearch:step}

But when to stop the iteration?
The iteration could be stopped for a sufficiently small value of the gradient (e.g.\  it's components or it's norm)\cite[89]{nonLinOptimerung}.
We decided to use a stopping condition onto the norm of the gradient \(\norm{\grad f} < \epsilon\).
The parameter \(\epsilon = \num{1d-5}\) was chosen for this condition in our implementation but could off course be overridden when using the implementation.

The implementation will feature some more aspects extracted into additional functions.
These are needed to recompute the stopping condition or to keep track of the visited points in parameter space.
The last one is used to visualize the behaviour of the algorithm.
Furthermore, there is value view implemented for retrieval of the fitted parameters.
This implementation is oriented as the corresponding one in the \verb|iminuit| framework.


% TODO: this sentence does not make sense in the current context!
The BFGS algorithm is a so called quasi-newton algorithm.
So it could be necessary to talk first about the newton algorithm.

%! Author = dominikfischer
%! Date = 2026-02-11

\subsection{Newton Procedure}\label{ssec:procedure:newton:derivation}
The Newton procedure for numerical optimisation could be derived form some more general thoughts.
The analytical condition for a minimum is the existence of a critical point where the gradient would vanish.
Thus, we are looking for the roots of the gradient of our \enquote{cost function}\footnote{Meaning: the negative log-likelihood. Will often use the more general terminology cost function in the follwing.}.
% TODO: citation for this?
Roots could be numerically estimated by the Newton-Raphson procedure, a fixed-point iteration.
Usually this procedure could be derived from a taylor expansion up to (including) first order (in our case one should expand the gradient and thus the function to optimise until including second order)
For a multivariate function we could describe the expansion by
\begin{equation}
    g(\vec{x} + \vec{h}) \approx g(\vec{x}) + \vec{h}\cdotp\grad g(\vec{x})+\order{h^2}\label{eq:newton:gradient:expansion}
\end{equation}
This could be rephrased for the usage in an algorithm which iteratively improves an approximation:
\begin{equation}
    \Leftrightarrow g(\vec{x}_{n+1} \approx g(\vec{x}_\text{n}) + (\vec{x}_\text{n+1} - \vec{x}_\text{n})\cdot\grad g(\vec{x}_\text{n})) + \order{h^2}\label{eq:newton:gradient:expansion:iteration}
\end{equation}

Now we want to assume that \(\vec{x}_\text{n+1}\) is a root of \(g\).
In this case we find:
\begin{align}
    0 = \stackrel{!}{=}g(\vec{x}_\text{n+1}) &= g(\vec{x}_\text{n+}) + (\vec{x}_\text{n+1} - \vec{x}_\text{n})\cdot\grad g(\vec{x}_\text{n}))\\
    \Rightarrow \vec{x}_{\text{n+1}}\cdotp\gradient g(\vec{x}_\text{n}) &= g(\vec{x}_\text{n}) - \vec{x}_{\text{n}}\cdot\grad g(\vec{x}_\text{n})\\
    \Rightarrow \vec{x}_{\text{n+1}} &= \vec{x}_{\text{n}} - J_{\text{g}}^{-1}g(\vec{x}_\text{n})\label{eq:newqton:raphson:general:step}
\end{align}
In the last step \cref{eq:newqton:raphson:general:step} the gradient was replaced by the more general jacobi-matrix.

But our primary interest is not to find the roots of \(g = \grad f\), but to find the minimum of \(f\)
So we will need to use the taylor expansion of \(f\) up to second order\cite{nonLinOptimerung,Nocedal}:
\begin{equation}
    f(\vec{x} + \vec{h}) \approx f(\vec{x}) + J_{\text{f}}(\vec{x})\vec{h} + \frac{1}{2}\vec{h}^{\text{T}}H_{\text{f}}\vec{h}+\order{h^3}\label{eq:newton:cost:expansion}
\end{equation}
From this follows immediately the Newton iteration for numerical optimisation\cite[91]{nonLinOptimerung}
\begin{equation}
    \vec{x}_{\text{n+1}} = \vec{x}_{\text{n}} -\alpha^{(k)} H_{\text{f}}^{-1}\cdot\grad f(\vec{x}_\text{n})\label{eq:newton:iteration}
\end{equation}
For investigations of the convergence behaviour\cite{nonLinOptimerung, Nocedal}.
But it could not be guaranteed that the final result would be a minimum it could as well be a maximum as this procedure is not able to distinguish between maxima and minima of a function.
Also, the found optimum is always just a local one and not necessarily a global.
A major drawback of the newton procedure is that the calculation and inversion of the hesse matrix is computational expensive.

\paragraph{Implementation notes}
Within the code base also a implementation of the Newton procedure is provided.
This is fitted into the skeleton of the bfgs procedure with uses a function to update the hessian.
This update function is also used but with a different implementation.
It is not a \enquote{update} but a full recomputation of the hessian with subsequently performed matrix inversion by LU-decomposition\footnote{The implementation of the LU decomposition without pivoting follows the \enquote{Computerphysik} lecture of the bachelor.}.




\subsection{BFGS}\label{ssec:bfgs}
% TODO: redistribute this parts along the different files such that the detailed version could be changed simply by commenting here.
%! Author = dominikfischer
%! Date = 2026-02-10

\subsubsection{General aspects of BFGS}
Considering the computational costs of the Newton algorithm, it is much more desirable to not calculate the exact hessian (or it's inverse) on every step of the iteration, but (calculate) only an approximation.
The approximation must be chosen such that the convergence rate is not reduced too much\cite[176]{nonLinOptimerung}.
This is exactly what is done by the BFGS algorithm.
In case of the BFGS algorithm the current approximation of the (inverse) hessian is updated on each iterative step by using the newly gained information from this step.
But, the BFGS update strategy needs to be seeded once by a \textbf{positive definite} matrix to start with.
How, to choose such a seed?
In the simplest case one could simply use the identity matrix for this task.
To speed up convergence, we could seed the update strategy on first iteration by the exact inverse Hessian\footnote{This is still to be calculated numerically.}
% TODO: fix this; I'm not sure whether this citation is correct!
This last suggestion should have convergence properties similar to the Newton-Raphson procedure, at much lower computational costs\cite{Nocedal, nonLinOptimerung}.
% TODO: shortcut code for the bfgs derivation?
%! Author = dominikfischer
%! Date = 2026-02-10

\subsubsection{The Hessian matrix}
% TODO: some parts of this paragraph needs to used anyway;
The BFGS update strategy could be derived from the Newton procedure \cref{eq:newton:iteration}.
This could be transformed such that we arrive at the from \(B_\text{k}\cdot p_\text{k} = -\grad f(x_\text{k})\).
Here \(B_\text{k}\) is not the hessian but an approximation to it and \(p_\text{k}\) is the search direction of the current iteration step.
These formulas for the step \(k\) and the step \(k+1\) are not independent.
% update accounts for the curvature at the current iteration
So it is possible to derive the secant equation\footnote{Also known as the Quasi-Newton equation.}\cite[175-216]{nonLinOptimerung}\cite[136-167]{Nocedal}:
\begin{align}
    B_\text{k+1} (x_\text{k+1} - x_\text{k}) &= \grad f(x_\text{k+1}) - \grad f(x_\text{k})\nonumber\\
    \Leftrightarrow B_{\text{k+1}}\cdot\vec{s}_{\text{k}} &= \vec{y}_{\text{k}}\label{eq:bfgs:secant}\\
    \vec{s}_{\text{k}} &= \vec{x}_{\text{k+1}}-\vec{x}_{\text{k}}\nonumber\\
    \vec{y}_{\text{k}} &= \grad f(\vec{x}_\text{k+1}) - \grad f(\vec{x}_\text{k})\nonumber
\end{align}
This could also be seen by using the taylor expansion of \(\grad f\) or \(f\).
As the Hessian could be defined as following \((D\grad f)_\text{i} = \nabla((\nabla f)_\text{i})\rightarrow H_\text{f}\).
If we drop the requirement for positive definiteness of the approximated hessian matrix, we could use the so-called Rank 1 update.
This means that we add to a previous approximation of the hessian matrices of rank 1\cite[176-216]{nonLinOptimerung}.
But we will try to use two (symmetric) rank 1 matrices which results in a rank 2 update strategy.
This update strategy will only preserve the positive definiteness of the Hessian if the curvature condition\(\vec{s}_\text{k}^\text{T}\cdot\vec{y}_\text{k} = \left\langle \vec{s}_\text{k}, \vec{y}_\text{k} \right\rangle > 0\) is fulfilled\cite[136-167]{Nocedal}\footnote{Would always be satisfied for a \textbf{convex} function.}.
Therefore, we must require at least Wolfe conditions (see \cref{ssec:impl:line})
The update formula would look as follows:
\begin{equation}
    B_{\text{k+1}} = B_{\text{k}} + \alpha\cdot\vec{u}\vec{u}^{\text{T}} + \beta\cdot\vec{v}\vec{v}^{\text{T}}\label{eq:bfgs:rank:update}
\end{equation}
But how to determine the vectors \(\vec{u}\) and \(\vec{v}\) as well as the coefficients in \cref{eq:bfgs:rank:update}?\footnote{Please note that the product of a vector and it's transposed is to be understand as the outer product of the vector with itself and not as an inner product}
It is as simple as applying the secant equation \cref{eq:bfgs:secant} to it.\footnote{I won't recite all the steps in between here as this would overreach the page limit definitely.}
% comment this coefficients derivation out if no longer needed
%! Author = dominikfischer
%! Date = 2026-02-11
% contains additional latex code for the full derivation of the bfgs update strategy which should be left out
Upfront it is necessary to make a particular choice to be reasoned later.
For the rank 1 update we will choose \(\vec{u} = \vec{y}_\text{k}\) and \(\vec{v} = B_\text{k}\vec{s}_\text{k}\).
We will here apply the secant equations to \textbf{both} sides of \cref{eq:bfgs:rank:update} where we will write the outer product of the vectors directly as a matrix.
\begin{align}
    B_{\text{k+1}}\vec{s}_{\text{k}} &= \vec{y}_\text{k}\nonumber\\
    B_\text{k}\vec{s}_\text{k} + U_{\text{k}}\vec{s}_{\text{k}} + V_{\text{k}}\vec{s}_{\text{k}} &= \vec{y}_{\text{k}}\nonumber\\
    &= \vec{v}(1+\beta\vec{v}^\text{T}\vec{s}_\text{k}) + \alpha\vec{u}\vec{u}^{\text{T}}\vec{s}_{\text{k}}\nonumber\\
    \Rightarrow \vec{v}(1+\beta\vec{v}^\text{T}\vec{s}_\text{k}) = \vec{u}(1-\alpha\vec{u}^\text{T}\vec{s}_\text{k}) = \text{const} = 0\nonumber\\
    \beta\vec{v}^{\text{T}}\vec{s}_{\text{k}} = -1 &\wedge 1 = \alpha\vec{v}^{\text{T}}\vec{s}_{\text{k}}\nonumber\\
    \Rightarrow \beta = -\frac{1}{\vec{s}_\text{k}^\text{T}B_\text{k}\vec{s}_\text{k}} &\wedge \alpha = \frac{1}{\vec{y}_\text{k}^\text{T}\vec{s}_\text{k}}\nonumber
\end{align}
In the last step it was also used that w.r.t.\ Schwartz theorem for partial derivatives the hessian matrix should be symmetric.\footnote{This calculation could also been done with a constant other than 0 but would be more complicated then.}

So as the final result we get for these quantities:
\begin{align}
    \vec{u} &= \vec{y}_\text{k}\nonumber\\
    \vec{v} &= B_\text{k}\vec{s}_\text{k}\nonumber\\
    \beta &= -\frac{1}{\vec{s}_\text{k}^\text{T}B_\text{k}\vec{s}_\text{k}}\nonumber\\
    \alpha &= \frac{1}{\vec{y}_\text{k}^\text{T}\vec{s}_\text{k}}\nonumber
\end{align}
Now apply this onto the update formula \cref{eq:bfgs:rank:update}
\begin{equation}
    B_{\text{k+1}} = B_{\text{k}} + \frac{\vec{y}_\text{k}\vec{y}_\text{k}^\text{T}}{\vec{y}_\text{k}^\text{T}\vec{s}_\text{k}} - \frac{B_\text{k}\vec{s}_\text{k}\vec{s}_\text{k}^\text{T}B_\text{k}}{\vec{s}_\text{k}^\text{T}B_\text{k}\vec{s}_\text{k}}\label{eq:bfgs:strategy}
\end{equation}
% FIXME: this sentence seems to make no sense.
But this formula could only be used for the update step.

\subsubsection{The inverse Hessian}
For the gradient descent iteration the inverse hessian is required and not the hessian itself.\footnote{Instead of using the inverse the hessian could be used and the search direction is then to be estimated by solving a linear equation system\cite{nonLinOptimerung, Nocedal}.}
Fortunately, it is possible to approximate the inverse hessian by a similar update formula.
To achieve this the \enquote{Sherman-Morisson-Woodburry-Formula}\cite[176-216]{nonLinOptimerung}\cite[136-167]{Nocedal} is to be used.
The meaning of this is:
\(A + \vec{v}\vec{v}^\text{T}\) is invertible if and only if \(1 + \vec{v}^\text{T}A^{-1}\vec{v}\neq 0\).
In this case (which is fulfilled for our update formula) we can use for the inverse:
\begin{equation}
    (A + \vec{v}\vec{v}^\text{T})^{-1} = A^{-1} - \frac{A^{-1}\vec{v}\vec{v}^\text{T}A^{-1}}{1+\vec{v}^\text{T}A^{-1}\vec{v}}\label{eq:sherman}
\end{equation}
This formula could be applied onto our rank 1 update \cref{eq:bfgs:strategy}
% comment this out and give only the final result if necessary?
%! Author = dominikfischer
%! Date = 2026-02-11
% Further latex code for the inversion of the hessian update formula!
\begin{align}
    B_\text{k+1}^{-1} &= \left( \Identity - \frac{\vec{s}_\text{k}\vec{y}_\text{k}^\text{T}}{\bfgsNormFactor}\right)B_\text{k}^{-1}\left( \Identity - \frac{\vec{y}_\text{k}\vec{s}_\text{k}^\text{T}}{\bfgsNormFactor}\right) + \frac{\vec{s}_\text{k}\vec{s}_\text{k}^\text{T}}{\bfgsNormFactor}|\text{expand}\nonumber\\
    &= \left( \Identity - \frac{\vec{s}_\text{k}\vec{y}_\text{k}^\text{T}}{\bfgsNormFactor}\right)\left( B_\text{k}^{-1} - \frac{B_\text{k}^{-1}\vec{y}_\text{k}\vec{s}_\text{k}^\text{T}}{\bfgsNormFactor}\right) + \frac{\vec{s}_\text{k}\vec{s}_\text{k}^\text{T}}{\bfgsNormFactor}\nonumber\\
    &= B_\text{k}^{-1} - \frac{B_\text{k}^{-1}\vec{y}_\text{k}\vec{s}_\text{k}^\text{T}}{\bfgsNormFactor} - \frac{\vec{s}_\text{k}\vec{y}_\text{k}^\text{T}B_\text{k}^{-1}}{\bfgsNormFactor} + \frac{\vec{s}_\text{k}\vec{y}_\text{k}^\text{T}B_\text{k}^{-1}\vec{y}_\text{k}\vec{s}_\text{k}^\text{T}}{\bfgsNormFactor^2} + \frac{\vec{s}_\text{k}\vec{s}_\text{k}^\text{T}}{\bfgsNormFactor}\nonumber\\
    &= B_\text{k}^{-1} + \frac{\vec{s}_\text{k}\vec{y}_\text{k}^\text{T}B_\text{k}^{-1}\vec{y}_\text{k}\vec{s}_\text{k}^\text{T}}{\bfgsNormFactor^2} + \frac{\vec{s}_\text{k}\vec{s}_\text{k}^\text{T} - B_\text{k}^{-1}\vec{y}_\text{k}\vec{s}_\text{k}^\text{T} - \vec{s}_\text{k}\vec{y}_\text{k}^\text{T}B_\text{k}^{-1}}{\bfgsNormFactor}|\text{symmetric}\nonumber\\
    &= B_{\text{k}}^{-1} + \frac{(\vec{y}_\text{k}^\text{T}B_\text{k}^{-1}\vec{y}_\text{k})(\vec{s}_\text{k}\vec{s}_\text{k}^\text{T}) + (\vec{y}_\text{k}^\text{T}\vec{s}_\text{k})(\vec{s}_\text{k}\vec{s}_\text{k}^\text{T})}{\bfgsNormFactor^2} - \frac{B_\text{k}^{-1}\vec{y}_\text{k}\vec{s}_\text{k}^\text{T} + \vec{s}_\text{k}\vec{y}_\text{k}^\text{T}B_\text{k}^\text{T}}{\bfgsNormFactor}\nonumber\\
    &= B_{\text{k}}^{-1} + \frac{(\vec{y}_\text{k}^\text{T}B_\text{k}^{-1}\vec{y}_\text{k} + \vec{y}_\text{k}^\text{T}\vec{s}_\text{k})(\vec{s}_\text{k}\vec{s}_\text{k}^\text{T})}{\bfgsNormFactor^2} - \frac{B_\text{k}^{-1}\vec{y}_\text{k}\vec{s}_\text{k}^\text{T} + \vec{s}_\text{k}\vec{y}_\text{k}^\text{T}B_\text{k}^\text{T}}{\bfgsNormFactor}
\end{align}

So we have for the final update formula\cite[176-216]{nonLinOptimerung}:
\begin{equation}
    B_{\text{k+1}}^{-1} = B_{\text{k}}^{-1} + \frac{(\vec{y}_\text{k}^\text{T}B_\text{k}^{-1}\vec{y}_\text{k} + \vec{y}_\text{k}^\text{T}\vec{s}_\text{k})(\vec{s}_\text{k}\vec{s}_\text{k}^\text{T})}{\bfgsNormFactor^2} - \frac{B_\text{k}^{-1}\vec{y}_\text{k}\vec{s}_\text{k}^\text{T} + \vec{s}_\text{k}\vec{y}_\text{k}^\text{T}B_\text{k}^\text{T}}{\bfgsNormFactor}\label{eq:bfgs:update:strategy}
\end{equation}





%! Author = dominikfischer
%! Date = 2026-02-27

\subsubsection{Implementation of BFGS}
The implementation of the bfgs is the implementation of the function which calculates the current search direction.
It is constructed such that it could accept an already computed gradient for the current point, e.g. from the run of the previous iterations line search.
The actual procedure will depend on the choice of the algorithm.
The \verb|newton| choice was already discussed a bit.
We will concentrate here onto the different bfgs choices, as the traditional gradient descent procedures were implemented but are not the goal of this project.
The algorithm will remeber the last computed hessian.
If no hessian is computed so far, e.g.\ it is the first iteration, then a seeding value of the hessian is required.
The different choices are as follows:\footnote{The framework minimizer function accepts a keyword 'algo', to specify the algorithm to use. We will give the keywords here with some explanation (At least some of the keywords)}
\begin{description}
    \item[bfgs simple] In this simplest possible seeding implementation, the inverse of the hessian is seeded by the identity of the corresponding dimension.
    \item[bfgs quasi] The \verb|bfgs quasi| implementation uses the an approximation by the reciprocals of the second order derivatives at the start point of the iteration to seed the inverse hessian.
    \item[bfgs] The \verb|bfgs| implementation will perform a full calculation of the hessian and invert it to seed the approximative inverse hesse matrix for the algorithm.
\end{description}



\subsection{Line search}\label{ssec:impl:line}
%! Author = dominikfischer
%! Date = 2026-02-21
\subsubsection{General aspects}
After finding suitable search direction and update strategies for these, we will need as a last step to estimate a suitable step for the iteration.
To find the optimum of the function \(f\) in the search direction a line search is performed.
An \textbf{exact} line search would be defined by \(\alpha_\text{k} = \arg\min f(\vec{x}_\text{k} + \alpha \vec{p}_\text{k})\).
But such an exact procedure could be computational expensive in particular if performed on every step.
So a tradeoff is necessary.
On the other hand, for our purpose the use of the Armijo condition and the (strong) Wolfe condition should be sufficient.\cite[135-162]{Nocedal}
The implementation of this iterative algorithm will be described in\cref{ssec:line:backtracking}.
In the next section we will first consider conditions to be fulfilled by an \enquote{optimal} step size.

\subsubsection{Conditions for a line search}
As we will perform an inexact line search we will need to discuss conditions which will be satisfied by the inexact optimum.
The simplest imaginable condition would be a significant decrease \(f(\vec{x}_\text{k} + \alpha^\text{(k)}\vec{p}_\text{k}) < f(\vec{x}_\text{k})\)\cite[32]{Nocedal}.
But this will not be enough to ensure convergence.
The following paragraphs will discuss well known conditions in use for line search algorithms and implemented for our own line search.

\paragraph{Armijo condition}
The Armijo condition should ensure a sufficient decrease of our function in the search direction.
Therefore define first the local slope \(m = (\gradient f(\vec{x}))^\text{T}\vec{p}\).
The decrease maybe controlled by a parameter \(c\).
For estimating the decrease use the taylor expansion of our function to be optimised\cite{Armijo}\cite[108-113]{nonLinOptimerung}\cite[33-36]{Nocedal}:
\begin{align}
    f(\vec{x} + \alpha\cdot\vec{p}) \approx f(\vec{x}) + \alpha\cdot\gradient f \vec{p} + \order{\alpha^2}\\
    \Rightarrow f(\vec{x})-f(\vec{x} + \alpha\cdot\vec{p}) = -(\gradient f\cdot\vec{p})\cdot\alpha = -\alpha m + \order{\alpha^2}\\
    \Rightarrow f(\vec{x})-f(\vec{x} + \alpha\cdot\vec{p}) \geq -\alpha c m\label{eq:linesearch:armijo}
\end{align}\footnote{The condition given bei Armijo uses the norm of the gradient, for the simplest implementation \(\vec{d} = -\grad f\) this equivalent to the stated condition.}
This condition won't be enought to ensure sufficient decrease as it could be satisfied for all sufficiently small values.
If the step size used gets to small the algorithm won't make any real process.
This leads finally to a condition on the curvature\cite[33-36]{Nocedal}.

\paragraph{Wolfe condition}
In Addition to the merely Armijo-Condition the (strong) Wolfe condition could be used.
% FIXME: I'm not sure whether this citation here is correctly. - Citation is correct but I originally meant to cite another book/paper
According to\cite{Nocedal} these will also make sure that the step size is not too small and acts like a lower bound.
The Wolfe condition is a requirement to the curvature along the line and minimising it\cite[117]{nonLinOptimerung}\cite[33-36]{Nocedal}:
\begin{equation}
    -\vec{p}_{\text{k}}^{\text{T}}\gradient f(\vec{x}_\text{k} + \alpha_\text{k}\vec{p}_\text{k}^\text{T}) \leq -c_2 \vec{p}_\text{k}^\text{T}\grad f(\vec{x}_\text{k}) = t_2\label{eq:linesearch:wolfe:condition}
\end{equation}
To fulfill both conditions at the same time it is necessary to have \(c_1 < c_2\) otherwise both conditions will contradict each other.
W.r.t.\cite{Nocedal} we'll use the values \(c_1 = 1e-4\) and \(c_2 = \num{0.9}\) for the parameters.
The Wolfe condition should ensure convergence of the gradient to zero.
To prevent the line search algorithm to choose a much to small step size it is also possible to improve the Wolfe-condition by using the so-called \enquote{strong Wolfe condition}\cite[119]{nonLinOptimerung}:
\begin{equation}
    \abs{\grad f(\vec{x}+\alpha\vec{p})^\text{T}\vec{p}} \leq -c_2\grad f(\vec{x})^{\text{T}}\vec{p}\label{eq:linesearch:condition:strongwolfe}
\end{equation}
This condition should exclude points that are far away from stationary points for our line search.

While trying to find a optimal step size the left-hand-side of\cref{eq:linesearch:wolfe:condition} gives some more information about the termination of this procedure\cite[33-36]{Nocedal}.
If it is only slightly negative or even positive, we could not achieve any further process by this search direction and the line search should be terminated.\footnote{This could also be seen as sign to terminate the whole optimisation procedure.}
% TODO: this here requires a citation
In particular when using the BFGS algorithm to determine the search direction the curvature condition\cref{eq:linesearch:wolfe:condition, eq:linesearch:condition:strongwolfe} should be used to ensure that the approximated inverse hessian remains positive definite by the update strategy.

\subsubsection{Improving the step size estimation}
% TODO: this improvement needs still to be incorporated into the implementation of the line search algorithm!
Reducing the current step size by certain factor/fraction is a quite simple mechanism to perform the estimation of the step size.
But we could also use the knowledge that the step size problem could be solved analytically for a quadratic function\cite[103]{nonLinOptimerung}\cite[42-44]{Nocedal}:
\begin{align}
    f(\vec{x}) &= \frac{1}{2}\vec{x}^{\text{T}}Q\vec{x}+\vec{b}^{\text{T}}\vec{x}\nonumber\\
    \alpha &= -\frac{(Q\overline{\vec{x}}+\vec{b})^\text{T}\vec{p}}{\vec{p}^\text{T}Q\vec{p}}\label{eq:perfect:step:size}
\end{align}
This could be used to seed the step size in the iteration of the approximate line search algorithm by approximating \(f\) locally by a quadratic function and apply this exact step size to this approximation of \(f\).
After such a seed, it will still be necessary to check for the provided conditions and perform the usual line search algorithm if necessary.\cite[120-123]{nonLinOptimerung}
The coefficients in the approximation of \(f\) are determined through the derivatives by mean of the taylor expansion.
These improving procedure are called \enquote{quadratic interpolation}.
The interpolation could be done according to Hermite\footnote{There also some other options for quadratic interpolation of the step size.} by defining the helper function \(h(t) = f(\vec{x}+t\cdot\vec{p})\) with the following Ansatz\cite[120-123]{nonLinOptimerung}.
\begin{align}
    p(t) &= k_0 + k_1\cdot (t-a) + k_2\cdot (t-a)^2\nonumber\\
    &= h(a) + h'(a)(t-a) + \frac{h(b)-h(a)-h'(a)(b-a)}{(b-a)^2}(t-a)^2
\end{align}
The step size for the global minimum of \(p(t)\) is now
\begin{equation}
    \alpha = a - \frac{h'(a)(b-a)^2}{2(h(b)-h(a))-h'(a)(b-a)}\label{eq:linesearch:impl:interpolation:qudratic}
\end{equation}
It is also possible to apply this interpolation procedure repeatedly\footnote{Not to be done; only used to seed a step size}\footnote{Also the use of a cubic interpolator is possible\cite{nonLinOptimerung, Nocedal}. Haben wir zwar implementiert wird aber nicht genutzt.}.
But the implementation must also use some safeguard in the case the interpolation fails.
% TODO: explanation of interpolations safeguard constructs is missing

\subsubsection{Backtracking Line search}\label{ssec:line:backtracking}
% TODO: What about the bracketing phase and the bisection phase mentioned in Nocedal?
To ensure algorithmically that the step size is neither to larger nor to small the backtracking line search algorithm is used.
In general the search will start with a (comparatively large) initial value for the step size and shrink the step size by \(\tau\in (0,1) \).
% TODO: description of the algorithm is still missing.
To start the line search we begin with an initial step size \(\alpha_0\) and limit the step size \(\alpha\) to a maximum value.
It could also be used a lower bound to not do an infinitely small step size (won't benefit from such a small step size)
To define the Armijo condition within the algorithm define \(t\equiv -c\cdot m\).
If the Armijo condition \cref{eq:linesearch:armijo} is already fulfilled for the seeding step size, we will increase the step size by \(1/\tau\) while the Armijo condition is satisfied and the maximum step size is not exceeded to prevent very small step sizes.
The substitution rule for the new step size is thus \(\frac{\alpha}{\tau}\rightarrow\alpha\)

In the other case where the Armijo-Condition is not satisfied, the step size is reduced by \(\tau\) until the condition is fulfilled.
To prevent the algorithm from consuming too much time on finding the approximate minimum along the line, the number of iterations is limited.
To improve this reduction of the step size, use an interval with upper and lower bounds which will be changed on each iteration, where the upper bound will be set on the currently tried step size if the Armijo condition is violated.
It is also possible to activate further conditions to check for.
% TODO: What about the usage of the (strong) Wolfe condition to enhance the line search?
% TODO: correct the line search implementation to use the (strong) wolfe condition!

After finding a step size which fulfills the Armijo condition, we will try to fulfill the Wolfe condition.
While the Wolfe condition is not satisfied increase the step size by \(\frac{\alpha}{\tau}\rightarrow\alpha\).
When intervals are used and the Wolfe condition is violated we will set a new lower bound to the currently used step size
As soon as the Wolfe condition is satisfied the iteration could be stopped and the step size is to be returned.
These procedures are in accordance with\cite{nonLinOptimerung}.

When a step a suitable step size is found, which fulfils all the required conditions the value is returned to the optimiser to perform the iteration step and start from begin with determining a new direction to search for the minimum of the log-likelihood.


% TODO: not sure whether this here is right place to mention it.
Nevertheless, this algorithm will still run into trouble if the search direction approaches a normal direction of the gradient.
In this case it won't be possible to make any real progress\cite{nonLinOptimerung, Nocedal}.
An possibility to solve this is to fallback to raw steepest descent implementation when approaching orthogonality as this cannot happen for \(\vec{p}_\text{k} = -\grad f(\vec{x}_\text{k})\).
If we could hold the search direction away from orthogonality then the Zoutendijk condition\cite[30-62]{Nocedal} implies
\begin{equation}
    \cos^2 \theta_{\text{k}}\norm{\grad f(\vec{x}_\text{k})}^2 \rightarrow 0\label{eq:zoutendijk}
\end{equation}
and therefore the convergence of optimisation algorithm.


    %! Author = dominikfischer
%! Date = 2026-02-10


\section{Results of optimisation tests}\label{sec:results}


\section{Conclusion}\label{sec:conclusion}
% Verzeichnisse für Abbildungen, Tabellen und Code-Beispiele
    \appendix
    \listoffigures
    \listoftables

%Literaturverzeichnis
    \printbibliography[heading=bibintoc]

\end{document}
