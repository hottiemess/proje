%! Author = dominikfischer
%! Date = 2026-02-10

\subsubsection{The Hessian matrix}
% TODO: some parts of this paragraph needs to used anyway;
The BFGS update strategy could be derived from the Newton procedure \cref{eq:newton:iteration}.
This could be transformed such that we arrive at the from \(B_\text{k}\cdot p_\text{k} = -\grad f(x_\text{k})\).
Here \(B_\text{k}\) is not the hessian but an approximation to it and \(p_\text{k}\) is the search direction of the current iteration step.
These formulas for the step \(k\) and the step \(k+1\) are not independent.
% update accounts for the curvature at the current iteration
So it is possible to derive the secant equation\footnote{Also known as the Quasi-Newton equation.}\cite[175-216]{nonLinOptimerung}\cite[136-167]{Nocedal}:
\begin{align}
    B_\text{k+1} (x_\text{k+1} - x_\text{k}) &= \grad f(x_\text{k+1}) - \grad f(x_\text{k})\nonumber\\
    \Leftrightarrow B_{\text{k+1}}\cdot\vec{s}_{\text{k}} &= \vec{y}_{\text{k}}\label{eq:bfgs:secant}\\
    \vec{s}_{\text{k}} &= \vec{x}_{\text{k+1}}-\vec{x}_{\text{k}}\nonumber\\
    \vec{y}_{\text{k}} &= \grad f(\vec{x}_\text{k+1}) - \grad f(\vec{x}_\text{k})\nonumber
\end{align}
This could also be seen by using the taylor expansion of \(\grad f\) or \(f\).
As the Hessian could be defined as following \((D\grad f)_\text{i} = \nabla((\nabla f)_\text{i})\rightarrow H_\text{f}\).
If we drop the requirement for positive definiteness of the approximated hessian matrix, we could use the so-called Rank 1 update.
This means that we add to a previous approximation of the hessian matrices of rank 1\cite[176-216]{nonLinOptimerung}.
But we will try to use two (symmetric) rank 1 matrices which results in a rank 2 update strategy.
This update strategy will only preserve the positive definiteness of the Hessian if the curvature condition\(\vec{s}_\text{k}^\text{T}\cdot\vec{y}_\text{k} = \left\langle \vec{s}_\text{k}, \vec{y}_\text{k} \right\rangle > 0\) is fulfilled\cite[136-167]{Nocedal}\footnote{Would always be satisfied for a \textbf{convex} function.}.
Therefore, we must require at least Wolfe conditions (see \cref{ssec:impl:line})
The update formula would look as follows:
\begin{equation}
    B_{\text{k+1}} = B_{\text{k}} + \alpha\cdot\vec{u}\vec{u}^{\text{T}} + \beta\cdot\vec{v}\vec{v}^{\text{T}}\label{eq:bfgs:rank:update}
\end{equation}
But how to determine the vectors \(\vec{u}\) and \(\vec{v}\) as well as the coefficients in \cref{eq:bfgs:rank:update}?\footnote{Please note that the product of a vector and it's transposed is to be understand as the outer product of the vector with itself and not as an inner product}
It is as simple as applying the secant equation \cref{eq:bfgs:secant} to it.\footnote{I won't recite all the steps in between here as this would overreach the page limit definitely.}
% comment this coefficients derivation out if no longer needed
%! Author = dominikfischer
%! Date = 2026-02-11
% contains additional latex code for the full derivation of the bfgs update strategy which should be left out
Upfront it is necessary to make a particular choice to be reasoned later.
For the rank 1 update we will choose \(\vec{u} = \vec{y}_\text{k}\) and \(\vec{v} = B_\text{k}\vec{s}_\text{k}\).
We will here apply the secant equations to \textbf{both} sides of \cref{eq:bfgs:rank:update} where we will write the outer product of the vectors directly as a matrix.
\begin{align}
    B_{\text{k+1}}\vec{s}_{\text{k}} &= \vec{y}_\text{k}\nonumber\\
    B_\text{k}\vec{s}_\text{k} + U_{\text{k}}\vec{s}_{\text{k}} + V_{\text{k}}\vec{s}_{\text{k}} &= \vec{y}_{\text{k}}\nonumber\\
    &= \vec{v}(1+\beta\vec{v}^\text{T}\vec{s}_\text{k}) + \alpha\vec{u}\vec{u}^{\text{T}}\vec{s}_{\text{k}}\nonumber\\
    \Rightarrow \vec{v}(1+\beta\vec{v}^\text{T}\vec{s}_\text{k}) = \vec{u}(1-\alpha\vec{u}^\text{T}\vec{s}_\text{k}) = \text{const} = 0\nonumber\\
    \beta\vec{v}^{\text{T}}\vec{s}_{\text{k}} = -1 &\wedge 1 = \alpha\vec{v}^{\text{T}}\vec{s}_{\text{k}}\nonumber\\
    \Rightarrow \beta = -\frac{1}{\vec{s}_\text{k}^\text{T}B_\text{k}\vec{s}_\text{k}} &\wedge \alpha = \frac{1}{\vec{y}_\text{k}^\text{T}\vec{s}_\text{k}}\nonumber
\end{align}
In the last step it was also used that w.r.t.\ Schwartz theorem for partial derivatives the hessian matrix should be symmetric.\footnote{This calculation could also been done with a constant other than 0 but would be more complicated then.}

So as the final result we get for these quantities:
\begin{align}
    \vec{u} &= \vec{y}_\text{k}\nonumber\\
    \vec{v} &= B_\text{k}\vec{s}_\text{k}\nonumber\\
    \beta &= -\frac{1}{\vec{s}_\text{k}^\text{T}B_\text{k}\vec{s}_\text{k}}\nonumber\\
    \alpha &= \frac{1}{\vec{y}_\text{k}^\text{T}\vec{s}_\text{k}}\nonumber
\end{align}
Now apply this onto the update formula \cref{eq:bfgs:rank:update}
\begin{equation}
    B_{\text{k+1}} = B_{\text{k}} + \frac{\vec{y}_\text{k}\vec{y}_\text{k}^\text{T}}{\vec{y}_\text{k}^\text{T}\vec{s}_\text{k}} - \frac{B_\text{k}\vec{s}_\text{k}\vec{s}_\text{k}^\text{T}B_\text{k}}{\vec{s}_\text{k}^\text{T}B_\text{k}\vec{s}_\text{k}}\label{eq:bfgs:strategy}
\end{equation}
% FIXME: this sentence seems to make no sense.
But this formula could only be used for the update step.

\subsubsection{The inverse Hessian}
For the gradient descent iteration the inverse hessian is required and not the hessian itself.\footnote{Instead of using the inverse the hessian could be used and the search direction is then to be estimated by solving a linear equation system\cite{nonLinOptimerung, Nocedal}.}
Fortunately, it is possible to approximate the inverse hessian by a similar update formula.
To achieve this the \enquote{Sherman-Morisson-Woodburry-Formula}\cite[176-216]{nonLinOptimerung}\cite[136-167]{Nocedal} is to be used.
The meaning of this is:
\(A + \vec{v}\vec{v}^\text{T}\) is invertible if and only if \(1 + \vec{v}^\text{T}A^{-1}\vec{v}\neq 0\).
In this case (which is fulfilled for our update formula) we can use for the inverse:
\begin{equation}
    (A + \vec{v}\vec{v}^\text{T})^{-1} = A^{-1} - \frac{A^{-1}\vec{v}\vec{v}^\text{T}A^{-1}}{1+\vec{v}^\text{T}A^{-1}\vec{v}}\label{eq:sherman}
\end{equation}
This formula could be applied onto our rank 1 update \cref{eq:bfgs:strategy}
% comment this out and give only the final result if necessary?
%! Author = dominikfischer
%! Date = 2026-02-11
% Further latex code for the inversion of the hessian update formula!
\begin{align}
    B_\text{k+1}^{-1} &= \left( \Identity - \frac{\vec{s}_\text{k}\vec{y}_\text{k}^\text{T}}{\bfgsNormFactor}\right)B_\text{k}^{-1}\left( \Identity - \frac{\vec{y}_\text{k}\vec{s}_\text{k}^\text{T}}{\bfgsNormFactor}\right) + \frac{\vec{s}_\text{k}\vec{s}_\text{k}^\text{T}}{\bfgsNormFactor}|\text{expand}\nonumber\\
    &= \left( \Identity - \frac{\vec{s}_\text{k}\vec{y}_\text{k}^\text{T}}{\bfgsNormFactor}\right)\left( B_\text{k}^{-1} - \frac{B_\text{k}^{-1}\vec{y}_\text{k}\vec{s}_\text{k}^\text{T}}{\bfgsNormFactor}\right) + \frac{\vec{s}_\text{k}\vec{s}_\text{k}^\text{T}}{\bfgsNormFactor}\nonumber\\
    &= B_\text{k}^{-1} - \frac{B_\text{k}^{-1}\vec{y}_\text{k}\vec{s}_\text{k}^\text{T}}{\bfgsNormFactor} - \frac{\vec{s}_\text{k}\vec{y}_\text{k}^\text{T}B_\text{k}^{-1}}{\bfgsNormFactor} + \frac{\vec{s}_\text{k}\vec{y}_\text{k}^\text{T}B_\text{k}^{-1}\vec{y}_\text{k}\vec{s}_\text{k}^\text{T}}{\bfgsNormFactor^2} + \frac{\vec{s}_\text{k}\vec{s}_\text{k}^\text{T}}{\bfgsNormFactor}\nonumber\\
    &= B_\text{k}^{-1} + \frac{\vec{s}_\text{k}\vec{y}_\text{k}^\text{T}B_\text{k}^{-1}\vec{y}_\text{k}\vec{s}_\text{k}^\text{T}}{\bfgsNormFactor^2} + \frac{\vec{s}_\text{k}\vec{s}_\text{k}^\text{T} - B_\text{k}^{-1}\vec{y}_\text{k}\vec{s}_\text{k}^\text{T} - \vec{s}_\text{k}\vec{y}_\text{k}^\text{T}B_\text{k}^{-1}}{\bfgsNormFactor}|\text{symmetric}\nonumber\\
    &= B_{\text{k}}^{-1} + \frac{(\vec{y}_\text{k}^\text{T}B_\text{k}^{-1}\vec{y}_\text{k})(\vec{s}_\text{k}\vec{s}_\text{k}^\text{T}) + (\vec{y}_\text{k}^\text{T}\vec{s}_\text{k})(\vec{s}_\text{k}\vec{s}_\text{k}^\text{T})}{\bfgsNormFactor^2} - \frac{B_\text{k}^{-1}\vec{y}_\text{k}\vec{s}_\text{k}^\text{T} + \vec{s}_\text{k}\vec{y}_\text{k}^\text{T}B_\text{k}^\text{T}}{\bfgsNormFactor}\nonumber\\
    &= B_{\text{k}}^{-1} + \frac{(\vec{y}_\text{k}^\text{T}B_\text{k}^{-1}\vec{y}_\text{k} + \vec{y}_\text{k}^\text{T}\vec{s}_\text{k})(\vec{s}_\text{k}\vec{s}_\text{k}^\text{T})}{\bfgsNormFactor^2} - \frac{B_\text{k}^{-1}\vec{y}_\text{k}\vec{s}_\text{k}^\text{T} + \vec{s}_\text{k}\vec{y}_\text{k}^\text{T}B_\text{k}^\text{T}}{\bfgsNormFactor}
\end{align}

So we have for the final update formula\cite[176-216]{nonLinOptimerung}:
\begin{equation}
    B_{\text{k+1}}^{-1} = B_{\text{k}}^{-1} + \frac{(\vec{y}_\text{k}^\text{T}B_\text{k}^{-1}\vec{y}_\text{k} + \vec{y}_\text{k}^\text{T}\vec{s}_\text{k})(\vec{s}_\text{k}\vec{s}_\text{k}^\text{T})}{\bfgsNormFactor^2} - \frac{B_\text{k}^{-1}\vec{y}_\text{k}\vec{s}_\text{k}^\text{T} + \vec{s}_\text{k}\vec{y}_\text{k}^\text{T}B_\text{k}^\text{T}}{\bfgsNormFactor}\label{eq:bfgs:update:strategy}
\end{equation}




